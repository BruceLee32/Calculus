\documentclass[working]{tuftebook}

\usepackage{standalone}
\input{../preamble/preamble.tex}
\input{../preamble/laterPreamble.tex}

\begin{document}
\chapter{Taylor's Theorem and Power Series}
\vspace{-2em}
\begin{marginfigure}
    \centering
    \incfig{bell-curve}
    \caption{A bell curve}
    \label{fig:bell-curve}
\end{marginfigure}

Some functions are impossible to integrate with elementary techniques. Some examples are
\[
    \frac{ \sin x}{x},\quad e^{-x^2},\,\text{and}\quad \sqrt{1-k^2sin^2x}
\]
Where $k$ is some constant less than 1. The first function is used as a common example to demonstrate the \emph{Feynman Technique}. The second function is a bell curve (see Figure \ref{fig:bell-curve}), which is very important in statistics. The third function comes up in trying to find the arc length of an ellipse. This is one of the reasons why there is a need to simplify these functions intomuch simpler ones so that we can easily integrate them. 

Another use is for calculating the values for special functions, such as the trigonometry functions. That is, calculating values such as $ \sin56 ^{\circ}$, for example.

\section{Approximating Functions using Polynomials}
One way to approximate functions is by using polynomials. 
\begin{theorem}[Taylor's Theorem]
    Any function satisfying certain conditions may be represented by Taylor Series
    \[
        f(x)=f(a)+f'(a)(x-a)+ \frac{f''(a)}{2!}(x-a)+...+ \frac{f^{(n)}(a)}{n!}(x-a)^n
    \]
\end{theorem}

\end{document}
