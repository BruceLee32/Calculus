\documentclass[working]{tuftebook}

\usepackage{standalone}
\input{../preamble/preamble.tex}
\input{../preamble/laterPreamble.tex}

\begin{document}
\chapter{DNE - Does Not Exist}
\vspace{-2em}
Some functions do not have a derivative at a certain point. The reason for this in most cases as we will see is because the slope $f'$ approaches $\infty$, $-\infty$, or simply doesn't exist.

Some common functions include $x^{ \frac{2}{3}}$ at $x=0$, $x^ \frac{1}{3}$ at $x=0$, $|x|$ at $x=0$, and $ \frac{1}{x}$ at $x=0$.

\begin{definition}[DNE]\label{def:DNE}
    The derivative of $f(x)$ at $x=a$ is considered DNE if
    \[
        \lim_{x\to a^+}f'(x)\neq \lim_{x\to a^-}f'(x)
    \]
\end{definition}

This definition is actually expanded upon in first to second year university mathematics \sidenote{We will refer to first year university mathematics at U1 mathematics. This applies to any year as well (example year 2 university mathematics is U2 mathematics).} by the \emph{Epsilon Delta Definition of a Limit}. Regardless, it is pretty intuitive that Definition \ref{def:DNE} is true.

\begin{marginfigure}
    \centering
    \incfig{one-over-x}
    \caption{Graph of $ f(x)= \frac{1}{x}$. There is a V.A at $x=0$.}
    \label{fig:one-over-x}
\end{marginfigure}

\begin{eg}
Consider the derivative of $f(x)= \frac{1}{x}$ at $x=0$ (see Figure \ref{fig:one-over-x}). We can immediately see that the value of $f'(x)$ at $x=a$ is unclear. To prove this, differentiate to get $f'(x)=- \frac{1}{x^2}$
\begin{align*}
    \lim_{x\to 0^+}&=-\infty\\ 
    \lim_{x\to 0^-}&= \infty 
\end{align*}
Which we can tell from the graph as well. Therefore, we see that we cannot reach a consenus.\sidenote{It should be noted that this was a bad example, but one that first comes to mind. The reason for this is because even if there weren't two values for $\displaystyle \lim_{x\to 0^+}f'(x)$ and $\displaystyle \lim_{x\to 0^-}f'(x)$, it still wouldn't have mattered, since they both evaluate to $\pm\infty$, which is DNE. However, I hope that it proves the point that if there are two possible values for the limiting case, then the derivative is defined as DNE.}
\end{eg}

\begin{marginfigure}
    \centering
    \incfig{x-cubed}
    \caption{Graph of $f(x)=x^ \frac{1}{3}$. There is a vertical POI at $x=0$.}
    \label{fig:x-cubed}
\end{marginfigure}

\begin{eg}
    Consider the derivative of $f(x)=x^3$ at $x=0$ (see Figure \ref{fig:x-cubed}). We can immediately see $\displaystyle \lim_{x\to 0}f'(x)=\infty$, implying DNE. 
\end{eg}

\begin{marginfigure}
    \centering
    \incfig{absolute-value-of-x}
    \caption{Graph of $f(x)= |x|$. The sharp turn at $x=0$ is what we call a \textbf{cusp}.}
    \label{fig:absolute-value-of-x}
\end{marginfigure}

\begin{eg}
    Consider the derivative of $f(x)=|x|$ at $x=0$ (see Figure \ref{fig:absolute-value-of-x}). We apply Definition \ref{def:DNE} to prove that $\displaystyle \lim_{x\to 0}f'(x)= \text{DNE}$. The derivative of $f(x)=|x|$ is interestingly $f'(x)= \frac{|x|}{x}$ or $f'(x)= \frac{x}{|x|}$. This implies
    \begin{align*}
        \lim_{x\to 0^+}f'(x)&=1\\
        \lim_{x\to 0^-}f'(x)&=-1
    \end{align*}
    Or you can just look at the graph to determine these values. Therefore, according to Definition \ref{def:DNE}, $f'(0)$ is undefined. 
\end{eg}

\begin{proposition}
    The derivative of $f'(x)$ at $x=a$ is DNE if: 
    \begin{enumerate}
    \setlength\itemsep{0.5em}
        \item{ $\displaystyle \lim_{x\to a^+}f'(x)\neq \lim_{x\to a^-}f'(x)$}
        \item{There is a horizontal POI at $x=a$.}
        \item{There is a cusp at $x=a$.}
    \end{enumerate}
\end{proposition}

\end{document}
