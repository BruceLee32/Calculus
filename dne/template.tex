\documentclass[working]{tuftebook}

\usepackage{standalone}
\usepackage[utf8]{inputenc}
\usepackage[T1]{fontenc}
\usepackage{textcomp}

\usepackage{url}

\usepackage[
    sorting=nyt,
    style=alphabetic
]{biblatex}
\addbibresource{bibliography.bib}

% switch to one below for APA 
% \usepackage[
    % sorting=nyt,
    % style=apa
% ]{biblatex}
% \addbibresource{bibliography.bib} % make bibliography.bib file -- formats are different (use google scholar)

\usepackage{hyperref}
\hypersetup{
    colorlinks,
    linkcolor={black},
    citecolor={black},
    urlcolor={blue!80!black}
}
\usepackage[noabbrev]{cleveref}

% Adds Bibliography, ... to Table of Contents
\usepackage[nottoc]{tocbibind}

\usepackage{graphicx}
\usepackage{float}
% \usepackage[usenames,dvipsnames,svgnames]{xcolor}

% \usepackage{cmbright}

\usepackage{amsmath, amsfonts, mathtools, amsthm, amssymb}
\usepackage{mathrsfs}
\usepackage{cancel}

\newcommand\N{\ensuremath{\mathbb{N}}}
\newcommand\R{\ensuremath{\mathbb{R}}}
\newcommand\Z{\ensuremath{\mathbb{Z}}}
\renewcommand\O{\ensuremath{\emptyset}}
\newcommand\Q{\ensuremath{\mathbb{Q}}}
\newcommand\C{\ensuremath{\mathbb{C}}}
\let\implies\Rightarrow
\let\impliedby\Leftarrow
\let\iff\Leftrightarrow
\let\epsilon\varepsilon

\usepackage{tikz}
\usepackage{tikz-cd}

% theorems
\usepackage{thmtools}
\usepackage{thm-restate}
\usepackage[framemethod=TikZ]{mdframed}
\mdfsetup{skipabove=1em,skipbelow=0em, innertopmargin=8pt, innerbottommargin=8pt}

\theoremstyle{definition}

\makeatletter

\declaretheoremstyle[headfont=\bfseries\sffamily, bodyfont=\normalfont, mdframed={ nobreak } ]{thmgreenbox}
\declaretheoremstyle[headfont=\bfseries\sffamily, bodyfont=\normalfont, mdframed={ nobreak } ]{thmredbox}
\declaretheoremstyle[headfont=\bfseries\sffamily, bodyfont=\normalfont]{thmbluebox}
\declaretheoremstyle[headfont=\bfseries\sffamily, bodyfont=\normalfont]{thmblueline}
\declaretheoremstyle[headfont=\bfseries\sffamily, bodyfont=\normalfont, numbered=no, mdframed={ rightline=false, topline=false, bottomline=false, }, qed=\qedsymbol ]{thmproofbox}
\declaretheoremstyle[headfont=\bfseries\sffamily, bodyfont=\normalfont, numbered=no, mdframed={ nobreak, rightline=false, topline=false, bottomline=false } ]{thmexplanationbox}

\declaretheoremstyle[headfont=\bfseries\sffamily, bodyfont=\normalfont, numbered=no, mdframed={ nobreak, rightline=false, topline=false, bottomline=false } ]{thmexplanationbox}


\declaretheorem[numberwithin=chapter, style=thmredbox, name=Definition]{definition}
\declaretheorem[sibling=definition, style=thmredbox, name=Corollary]{corollary}
\declaretheorem[sibling=definition, style=thmredbox, name=Proposition]{proposition}
\declaretheorem[sibling=definition, style=thmredbox, name=Theorem]{theorem}
\declaretheorem[sibling=definition, style=thmredbox, name=Lemma]{lemma}
\declaretheorem[sibling=definition, style=thmbluebox,  name=Example]{eg}
\declaretheorem[sibling=definition, style=thmbluebox,  name=Nonexample]{noneg}
\declaretheorem[sibling=definition, style=thmblueline, name=Remark]{remark}


\declaretheorem[numbered=no, style=thmbluebox,  name=Derivation]{derivation}
\declaretheorem[numbered=no, style=thmexplanationbox, name=Proof]{explanation}
\declaretheorem[numbered=no, style=thmproofbox, name=Proof]{replacementproof}
\declaretheorem[style=thmbluebox,  numbered=no, name=Exercise]{ex}
\declaretheorem[style=thmblueline, numbered=no, name=Note]{note}

% \renewenvironment{proof}[1][\proofname]{\begin{replacementproof}}{\end{replacementproof}}

% \AtEndEnvironment{eg}{\null\hfill$\diamond$}%

\newtheorem*{uovt}{UOVT}
\newtheorem*{notation}{Notation}
\newtheorem*{previouslyseen}{As previously seen}
\newtheorem*{problem}{Problem}
\newtheorem*{observe}{Observe}
\newtheorem*{property}{Property}
\newtheorem*{intuition}{Intuition}


\declaretheoremstyle[
    headfont=\bfseries\sffamily\color{RawSienna!70!black}, bodyfont=\normalfont,
    mdframed={
        linewidth=2pt,
        rightline=false, topline=false, bottomline=false,
        linecolor=RawSienna, backgroundcolor=RawSienna!5,
    }
]{todo}
\declaretheorem[numbered=no, style=todo, name=TODO]{TODO}


\usepackage{etoolbox}
\AtEndEnvironment{vb}{\null\hfill$\diamond$}%
\AtEndEnvironment{intermezzo}{\null\hfill$\diamond$}%

% http://tex.stackexchange.com/questions/22119/how-can-i-change-the-spacing-before-theorems-with-amsthm
% \def\thm@space@setup{%
%   \thm@preskip=\parskip \thm@postskip=0pt
% }

\usepackage{xifthen}

\makeatother

% figure support (https://castel.dev/post/lecture-notes-2)
\usepackage{import}
\usepackage{xifthen}
\pdfminorversion=7
\usepackage{pdfpages}
\usepackage{transparent}


\makeatletter
\newif\ifworking
\@ifclasswith{tuftebook}{working}{\workingtrue}{\workingfalse}
\makeatother

\newcommand{\incfig}[2][1]{%
    % \ifworking{\makebox[0pt][c]{\color{gray}{\scriptsize\textsf{#2}}}}\fi%
    \def\svgwidth{#1\textwidth}
    \import{../figures/}{#2.pdf_tex}
}

\newcommand{\fullwidthincfig}[2][0.90]{%
    % \ifworking{\makebox[0pt][l]{\color{gray}{\scriptsize\textsf{#2}}}}\fi%
    \def\svgwidth{#1\paperwidth}
    \import{../figures/}{#2.pdf_tex}
}



\newcommand{\minifig}[2]{%
    \def\svgwidth{#1}%
    \begingroup%
    \setbox0=\hbox{\import{../figures/}{#2.pdf_tex}}%
    \parbox{\wd0}{\box0}\endgroup%
    \hspace*{0.2cm}
}

% %http://tex.stackexchange.com/questions/76273/multiple-pdfs-with-page-group-included-in-a-single-page-warning
\pdfsuppresswarningpagegroup=1

\newcommand\todo[1]{\ifworking {{\color{red}{#1}}} \else {}\fi}
\newcommand\charlotte[1]{\ifworking {{\color{blue}{#1}}} \else {}\fi}

\author{Kenny Chen}

\usepackage{multirow}
\def\block(#1,#2)#3{\multicolumn{#2}{c}{\multirow{#1}{*}{$ #3 $}}}

% \overfullrule=1mm

\newenvironment{myproof}[1][\proofname]{%
  \proof[\rm \bf #1]%
}{\endproof}

% PERSONAL PREAMBLE 
\usepackage{physics} 

% Enumerate environments 
\newenvironment{2qu}
{
\begin{enumerate}[label=(\alph*)]
}
{\end{enumerate}}

\newenvironment{3qu}
{
\begin{enumerate}[label=(\roman*)]
}
{\end{enumerate}}

% Normal Environments 
\newenvironment{list0.5}
{
\begin{enumerate}
\setlength\itemsep{0.5em}
}
{\end{enumerate}}

% Problems Environment
\newenvironment{problems}
{
    \subsection{Problems}
    \begin{enumerate}
    \setlength\itemsep{1em}
        
}
{
\end{enumerate}
}

\newenvironment{solutions}
{
    \subsection{Solutions}
    \begin{enumerate}
    \setlength\itemsep{1em}
}
{
\end{enumerate}
}

\usepackage{pdfpages}

\usepackage{lipsum}
\usepackage{parskip}
\usepackage{titletoc}

\newcommand\circled[1]{
    \begin{tikzpicture}[baseline=(char.base)]%
        \node[circle,draw,inner sep=1pt] (char) {\textsf{#1}};%
\end{tikzpicture}}
% minicircle for in figures!
\newcommand\mc[1]{\footnotesize\circled{#1}}

\usepackage{cmbright}
\usepackage{bm}

% \usepackage{eso-pic}                % put things into background 
% \usepackage{lipsum}                 % for sample text

% \definecolor{reallylightgray}{HTML}{FAFAFA}
% \AddToShipoutPicture{% from package eso-pic: put something to the background
%     \ifthenelse{\isodd{\thepage}}{
%           % ODD page: left bar
%           \AtPageLowerLeft{% start the bar at the left bottom of the page
%             \put(\LenToUnit{\dimexpr\paperwidth-222pt},0){% move it to the top right
%                 \color{reallylightgray}\rule{222pt}{297mm}% }%
%           }%
%       }%
%       {%
%         \AtPageLowerLeft{% put it at the left bottom of the page
%           \color{reallylightgray}\rule{222pt}{297mm}%
%         }%
%    }%
% }



\begin{document}
\chapter{DNE - Does Not Exist}
\vspace{-2em}
Some functions do not have a derivative at a certain point. The reason for this in most cases as we will see is because the slope $f'$ approaches $\infty$, $-\infty$, or simply doesn't exist.

Some common functions include $x^{ \frac{2}{3}}$ at $x=0$, $x^ \frac{1}{3}$ at $x=0$, $|x|$ at $x=0$, and $ \frac{1}{x}$ at $x=0$.

\begin{definition}[DNE]\label{def:DNE}
    The derivative of $f(x)$ at $x=a$ is considered DNE if
    \[
        \lim_{x\to a^+}f'(x)\neq \lim_{x\to a^-}f'(x)
    \]
\end{definition}

This definition is actually expanded upon in first to second year university mathematics \sidenote{We will refer to first year university mathematics at U1 mathematics. This applies to any year as well (example year 2 university mathematics is U2 mathematics).} by the \emph{Epsilon Delta Definition of a Limit}. Regardless, it is pretty intuitive that Definition \ref{def:DNE} is true.

\begin{marginfigure}
    \centering
    \incfig{one-over-x}
    \caption{Graph of $ f(x)= \frac{1}{x}$. There is a V.A at $x=0$.}
    \label{fig:one-over-x}
\end{marginfigure}

\begin{eg}
Consider the derivative of $f(x)= \frac{1}{x}$ at $x=0$ (see Figure \ref{fig:one-over-x}). We can immediately see that the value of $f'(x)$ at $x=a$ is unclear. To prove this, differentiate to get $f'(x)=- \frac{1}{x^2}$
\begin{align*}
    \lim_{x\to 0^+}&=-\infty\\ 
    \lim_{x\to 0^-}&= \infty 
\end{align*}
Which we can tell from the graph as well. Therefore, we see that we cannot reach a consenus.\sidenote{It should be noted that this was a bad example, but one that first comes to mind. The reason for this is because even if there weren't two values for $\displaystyle \lim_{x\to 0^+}f'(x)$ and $\displaystyle \lim_{x\to 0^-}f'(x)$, it still wouldn't have mattered, since they both evaluate to $\pm\infty$, which is DNE. However, I hope that it proves the point that if there are two possible values for the limiting case, then the derivative is defined as DNE.}
\end{eg}

\begin{marginfigure}
    \centering
    \incfig{x-cubed}
    \caption{Graph of $f(x)=x^ \frac{1}{3}$. There is a vertical POI at $x=0$.}
    \label{fig:x-cubed}
\end{marginfigure}

\begin{eg}
    Consider the derivative of $f(x)=x^3$ at $x=0$ (see Figure \ref{fig:x-cubed}). We can immediately see $\displaystyle \lim_{x\to 0}f'(x)=\infty$, implying DNE. 
\end{eg}

\begin{marginfigure}
    \centering
    \incfig{absolute-value-of-x}
    \caption{Graph of $f(x)= |x|$. The sharp turn at $x=0$ is what we call a \textbf{cusp}.}
    \label{fig:absolute-value-of-x}
\end{marginfigure}

\begin{eg}
    Consider the derivative of $f(x)=|x|$ at $x=0$ (see Figure \ref{fig:absolute-value-of-x}). We apply Definition \ref{def:DNE} to prove that $\displaystyle \lim_{x\to 0}f'(x)= \text{DNE}$. The derivative of $f(x)=|x|$ is interestingly $f'(x)= \frac{|x|}{x}$ or $f'(x)= \frac{x}{|x|}$. This implies
    \begin{align*}
        \lim_{x\to 0^+}f'(x)&=1\\
        \lim_{x\to 0^-}f'(x)&=-1
    \end{align*}
    Or you can just look at the graph to determine these values. Therefore, according to Definition \ref{def:DNE}, $f'(0)$ is undefined. 
\end{eg}

\begin{proposition}
    The derivative of $f'(x)$ at $x=a$ is DNE if: 
    \begin{enumerate}
    \setlength\itemsep{0.5em}
        \item{ $\displaystyle \lim_{x\to a^+}f'(x)\neq \lim_{x\to a^-}f'(x)$}
        \item{There is a horizontal POI at $x=a$.}
        \item{There is a cusp at $x=a$.}
    \end{enumerate}
\end{proposition}

\end{document}
