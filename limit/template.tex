\documentclass[working]{tuftebook}

\usepackage{standalone}
\input{../preamble/preamble.tex}
\input{../preamble/laterPreamble.tex}

\begin{document}
\renewcommand{\thepage}{\arabic{page}}
\pagestyle{plain}
\pagestyle{normal}

\chapter{The Limit}
\vspace{-2em}
The limit is one of the most crucial topics in calculus, because it helps us define the derivative as well as calculate them. The limit is usually applied to a variable, where that variable approaches some number infinitely close.

There is generally one definition of the limit, the \emph{Epsilon Delta Definition of a Limit}, but that is too complicated for our purposes.\sidenote{The Epsilon Delta Definition of a Limit appears in year 1 to 2 of university calculus.} Our definition is much simpler.

\begin{definition}[Definition of a Limit]
    The limit for a variable $x$ to some value $a$ is denoted by 
    \[
        x\to a
    \]
    Or
    \[
        \lim_{x\to a}
    \]
    Where the lim stands for ``limit''. Both of these are pronounced ``as $x$ approaches $a$'', and that implies ``as $x$ becomes infinitely close to $a$''.
\end{definition}

What does it mean for $x$ to become ``infinitely close to $a$''? Let's consider for $a=2$. When we say $x\to 2$, this implies some number infinitely close to 2. This can be $1.0000....0001$ or $0.99999....9$; we can clearly see that both of these numbers are ``infinitely'' close to 2 (see Figure \ref{fig:infinitely-close}).  

\begin{figure}[ht]
    \centering
    \incfig{infinitely-close}
    \caption{We keep on dividing the parts between $ \frac{1}{2}$ and 1 by half until infinity. It will never reach 1, but it becomes infinitely close to 1 as we keep on repeating this process.}
    \label{fig:infinitely-close}
\end{figure}

The most important thing to remember for a limit is that it approaches the value, but is never that value. This becomes important when we are determing the derivative of functions using the First Principle.

It should be noted that the value for the limit of a variable is independent to other variables. For example, if we had $\displaystyle \lim_{x\to 0}(x+y)$, the $y$ is independent of the $x$, so we can really write $\displaystyle y+\lim_{x\to 0}x$.\sidenote{The limit rule that we just used is called the sum property for limits, where $y$ is a constant that isn't dependent of $x$.}

\section{Sum Property}\label{def:sum-property}
\begin{definition}
    For functions $f(x)$ and $g(x)$, the limit of their sum is the sum of their limits 
    \[
        \lim_{x\to a} \left[ f(x)+g(x) \right]= \lim_{x\to a}f(x)+ \lim_{x\to a}g(x)
    \]
    And if there is a constant $C$
    \[
        \lim_{x\to a} \left[ f(x)+C \right]= C+\lim_{x\to a}f(x)
    \]
\end{definition}
And we will not dive into the proof for this property. 

\section{Difference Property}
\begin{definition}
    For functions $f(x)$ and $g(x)$, the limit of their difference is the difference of their limits 
    \[
        \lim_{x\to a} \left[ f(x)-g(x) \right]= \lim_{x\to a}f(x)- \lim_{x\to a}g(x)
    \]
    And if there is a constant $C$
    \[
        \lim_{x\to a} \left[ f(x)-C \right]=-C+ \lim_{x\to a}f(x)
    \]
    Or 
    \[
        \lim_{x\to a} \left[ C-f(x) \right]=C- \lim_{x\to a}f(x)
    \]
\end{definition}
\begin{myproof}
    We will prove this using Definition \ref{def:sum-property}
    \begin{align*}
        \lim_{x\to a} \left[ f(x)-g(x) \right]&= \lim_{x\to a}f(x)+ (\lim_{x\to a} -\left[ g(x) \right])\\
                                              &= \lim_{x\to a}f(x)- \lim_{x\to a}g(x)
    \end{align*}
    Where it because the negative is independent of the limit (it is a constant multiplier), we can drag it outside of the limit.\sidenote{These proofs aren't ``mathematical'', but these properties are pretty intuitive so this will suffice. You can search up the actual proofs online. They are pretty weird though.}
\end{myproof}

\newpage
\section{Product Property}\label{def:product}
\begin{definition}
    For functions $f(x)$ and $g(x)$, the limit of their product is the product of their limits 
    \[
        \lim_{x\to a}f(x)g(x)= \lim_{x\to a}f(x)\cdot \lim_{x\to a}g(x)
    \]
\end{definition}
The proof for this is too obscure, so we won't include that here. Just remember this property.

\section{Quotient Property}
\begin{definition}
    For functions $f(x)$ and $g(x)$, the limit of their quotient is the quotient of their limits 
    \[
        \lim_{x\to a} \frac{f(x)}{g(x)}= \frac{ \lim_{x\to a}f(x)}{ \lim_{x\to a}g(x)}
    \]
\end{definition}
\begin{myproof}
    Using Definition \ref{def:product}
    \begin{align*}
        \lim_{x\to a} \frac{f(x)}{g(x)}&= \lim_{x\to a}f(x)[g(x)]^{-1}\\
                                       &= \lim_{x\to a}f(x) \lim_{x\to a}[g(x)]^{-1}\\ 
                                       &= \frac{ \lim_{x\to a}f(x)}{ \lim_{x\to a}g(x)}
    \end{align*}
\end{myproof}

\section{Using Limit Properties}
Now that we have established the properties of limits, we can use them in some examples. 

\begin{fullwidth}
\begin{eg}
    Evaluate $\displaystyle \lim_{h\to 0}(x+h)$. In this case, we see that the limit is evaluating $h\to0$. Therefore, $x$ is completely independent in this limit. Hence, we write
    \begin{align*}
        \lim_{h\to 0}(x+h)&=x+ \lim_{h\to 0}h\\ 
                          &=x
    \end{align*}
    Since as $h\to0$, we approximate it to be 0, leaving us with just $x$.
\end{eg}
\begin{eg}
    Evaluate $\displaystyle \lim_{x\to 2} \frac{x^2}{x+1}$. We can actually just plug in $x=2$ into the numerator and denominator 
    \begin{align*}
        \lim_{x\to 2} \frac{x^2}{x+1}&= \frac{(2)^2}{(2)+1}\\ 
                                     &= \frac{4}{3}
    \end{align*}
\end{eg}
\end{fullwidth}

\end{document}
