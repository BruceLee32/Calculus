\documentclass[working]{tuftebook}

\usepackage{standalone}
\usepackage[utf8]{inputenc}
\usepackage[T1]{fontenc}
\usepackage{textcomp}

\usepackage{url}

\usepackage[
    sorting=nyt,
    style=alphabetic
]{biblatex}
\addbibresource{bibliography.bib}

% switch to one below for APA 
% \usepackage[
    % sorting=nyt,
    % style=apa
% ]{biblatex}
% \addbibresource{bibliography.bib} % make bibliography.bib file -- formats are different (use google scholar)

\usepackage{hyperref}
\hypersetup{
    colorlinks,
    linkcolor={black},
    citecolor={black},
    urlcolor={blue!80!black}
}
\usepackage[noabbrev]{cleveref}

% Adds Bibliography, ... to Table of Contents
\usepackage[nottoc]{tocbibind}

\usepackage{graphicx}
\usepackage{float}
% \usepackage[usenames,dvipsnames,svgnames]{xcolor}

% \usepackage{cmbright}

\usepackage{amsmath, amsfonts, mathtools, amsthm, amssymb}
\usepackage{mathrsfs}
\usepackage{cancel}

\newcommand\N{\ensuremath{\mathbb{N}}}
\newcommand\R{\ensuremath{\mathbb{R}}}
\newcommand\Z{\ensuremath{\mathbb{Z}}}
\renewcommand\O{\ensuremath{\emptyset}}
\newcommand\Q{\ensuremath{\mathbb{Q}}}
\newcommand\C{\ensuremath{\mathbb{C}}}
\let\implies\Rightarrow
\let\impliedby\Leftarrow
\let\iff\Leftrightarrow
\let\epsilon\varepsilon

\usepackage{tikz}
\usepackage{tikz-cd}

% theorems
\usepackage{thmtools}
\usepackage{thm-restate}
\usepackage[framemethod=TikZ]{mdframed}
\mdfsetup{skipabove=1em,skipbelow=0em, innertopmargin=8pt, innerbottommargin=8pt}

\theoremstyle{definition}

\makeatletter

\declaretheoremstyle[headfont=\bfseries\sffamily, bodyfont=\normalfont, mdframed={ nobreak } ]{thmgreenbox}
\declaretheoremstyle[headfont=\bfseries\sffamily, bodyfont=\normalfont, mdframed={ nobreak } ]{thmredbox}
\declaretheoremstyle[headfont=\bfseries\sffamily, bodyfont=\normalfont]{thmbluebox}
\declaretheoremstyle[headfont=\bfseries\sffamily, bodyfont=\normalfont]{thmblueline}
\declaretheoremstyle[headfont=\bfseries\sffamily, bodyfont=\normalfont, numbered=no, mdframed={ rightline=false, topline=false, bottomline=false, }, qed=\qedsymbol ]{thmproofbox}
\declaretheoremstyle[headfont=\bfseries\sffamily, bodyfont=\normalfont, numbered=no, mdframed={ nobreak, rightline=false, topline=false, bottomline=false } ]{thmexplanationbox}

\declaretheoremstyle[headfont=\bfseries\sffamily, bodyfont=\normalfont, numbered=no, mdframed={ nobreak, rightline=false, topline=false, bottomline=false } ]{thmexplanationbox}


\declaretheorem[numberwithin=chapter, style=thmredbox, name=Definition]{definition}
\declaretheorem[sibling=definition, style=thmredbox, name=Corollary]{corollary}
\declaretheorem[sibling=definition, style=thmredbox, name=Proposition]{proposition}
\declaretheorem[sibling=definition, style=thmredbox, name=Theorem]{theorem}
\declaretheorem[sibling=definition, style=thmredbox, name=Lemma]{lemma}
\declaretheorem[sibling=definition, style=thmbluebox,  name=Example]{eg}
\declaretheorem[sibling=definition, style=thmbluebox,  name=Nonexample]{noneg}
\declaretheorem[sibling=definition, style=thmblueline, name=Remark]{remark}


\declaretheorem[numbered=no, style=thmbluebox,  name=Derivation]{derivation}
\declaretheorem[numbered=no, style=thmexplanationbox, name=Proof]{explanation}
\declaretheorem[numbered=no, style=thmproofbox, name=Proof]{replacementproof}
\declaretheorem[style=thmbluebox,  numbered=no, name=Exercise]{ex}
\declaretheorem[style=thmblueline, numbered=no, name=Note]{note}

% \renewenvironment{proof}[1][\proofname]{\begin{replacementproof}}{\end{replacementproof}}

% \AtEndEnvironment{eg}{\null\hfill$\diamond$}%

\newtheorem*{uovt}{UOVT}
\newtheorem*{notation}{Notation}
\newtheorem*{previouslyseen}{As previously seen}
\newtheorem*{problem}{Problem}
\newtheorem*{observe}{Observe}
\newtheorem*{property}{Property}
\newtheorem*{intuition}{Intuition}


\declaretheoremstyle[
    headfont=\bfseries\sffamily\color{RawSienna!70!black}, bodyfont=\normalfont,
    mdframed={
        linewidth=2pt,
        rightline=false, topline=false, bottomline=false,
        linecolor=RawSienna, backgroundcolor=RawSienna!5,
    }
]{todo}
\declaretheorem[numbered=no, style=todo, name=TODO]{TODO}


\usepackage{etoolbox}
\AtEndEnvironment{vb}{\null\hfill$\diamond$}%
\AtEndEnvironment{intermezzo}{\null\hfill$\diamond$}%

% http://tex.stackexchange.com/questions/22119/how-can-i-change-the-spacing-before-theorems-with-amsthm
% \def\thm@space@setup{%
%   \thm@preskip=\parskip \thm@postskip=0pt
% }

\usepackage{xifthen}

\makeatother

% figure support (https://castel.dev/post/lecture-notes-2)
\usepackage{import}
\usepackage{xifthen}
\pdfminorversion=7
\usepackage{pdfpages}
\usepackage{transparent}


\makeatletter
\newif\ifworking
\@ifclasswith{tuftebook}{working}{\workingtrue}{\workingfalse}
\makeatother

\newcommand{\incfig}[2][1]{%
    % \ifworking{\makebox[0pt][c]{\color{gray}{\scriptsize\textsf{#2}}}}\fi%
    \def\svgwidth{#1\textwidth}
    \import{../figures/}{#2.pdf_tex}
}

\newcommand{\fullwidthincfig}[2][0.90]{%
    % \ifworking{\makebox[0pt][l]{\color{gray}{\scriptsize\textsf{#2}}}}\fi%
    \def\svgwidth{#1\paperwidth}
    \import{../figures/}{#2.pdf_tex}
}



\newcommand{\minifig}[2]{%
    \def\svgwidth{#1}%
    \begingroup%
    \setbox0=\hbox{\import{../figures/}{#2.pdf_tex}}%
    \parbox{\wd0}{\box0}\endgroup%
    \hspace*{0.2cm}
}

% %http://tex.stackexchange.com/questions/76273/multiple-pdfs-with-page-group-included-in-a-single-page-warning
\pdfsuppresswarningpagegroup=1

\newcommand\todo[1]{\ifworking {{\color{red}{#1}}} \else {}\fi}
\newcommand\charlotte[1]{\ifworking {{\color{blue}{#1}}} \else {}\fi}

\author{Kenny Chen}

\usepackage{multirow}
\def\block(#1,#2)#3{\multicolumn{#2}{c}{\multirow{#1}{*}{$ #3 $}}}

% \overfullrule=1mm

\newenvironment{myproof}[1][\proofname]{%
  \proof[\rm \bf #1]%
}{\endproof}

% PERSONAL PREAMBLE 
\usepackage{physics} 

% Enumerate environments 
\newenvironment{2qu}
{
\begin{enumerate}[label=(\alph*)]
}
{\end{enumerate}}

\newenvironment{3qu}
{
\begin{enumerate}[label=(\roman*)]
}
{\end{enumerate}}

% Normal Environments 
\newenvironment{list0.5}
{
\begin{enumerate}
\setlength\itemsep{0.5em}
}
{\end{enumerate}}

% Problems Environment
\newenvironment{problems}
{
    \subsection{Problems}
    \begin{enumerate}
    \setlength\itemsep{1em}
        
}
{
\end{enumerate}
}

\newenvironment{solutions}
{
    \subsection{Solutions}
    \begin{enumerate}
    \setlength\itemsep{1em}
}
{
\end{enumerate}
}

\usepackage{pdfpages}

\usepackage{lipsum}
\usepackage{parskip}
\usepackage{titletoc}

\newcommand\circled[1]{
    \begin{tikzpicture}[baseline=(char.base)]%
        \node[circle,draw,inner sep=1pt] (char) {\textsf{#1}};%
\end{tikzpicture}}
% minicircle for in figures!
\newcommand\mc[1]{\footnotesize\circled{#1}}

\usepackage{cmbright}
\usepackage{bm}

% \usepackage{eso-pic}                % put things into background 
% \usepackage{lipsum}                 % for sample text

% \definecolor{reallylightgray}{HTML}{FAFAFA}
% \AddToShipoutPicture{% from package eso-pic: put something to the background
%     \ifthenelse{\isodd{\thepage}}{
%           % ODD page: left bar
%           \AtPageLowerLeft{% start the bar at the left bottom of the page
%             \put(\LenToUnit{\dimexpr\paperwidth-222pt},0){% move it to the top right
%                 \color{reallylightgray}\rule{222pt}{297mm}% }%
%           }%
%       }%
%       {%
%         \AtPageLowerLeft{% put it at the left bottom of the page
%           \color{reallylightgray}\rule{222pt}{297mm}%
%         }%
%    }%
% }



\begin{document}
\chapter{The Limit}
\vspace{-2em}
The limit is one of the most crucial topics in calculus, because it helps us define the derivative as well as calculate them. The limit is usually applied to a variable, where that variable approaches some number infinitely close.

There is generally one definition of the limit, the \emph{Epsilon Delta Definition of a Limit}, but that is too complicated for our purposes.\sidenote{The Epsilon Delta Definition of a Limit appears in year 1 to 2 of university calculus.} Our definition is much simpler.

\begin{definition}[Definition of a Limit]
    The limit for a variable $x$ to some value $a$ is denoted by 
    \[
        x\to a
    \]
    Or
    \[
        \lim_{x\to a}
    \]
    Where the lim stands for ``limit''. Both of these are pronounced ``as $x$ approaches $a$'', and that implies ``as $x$ becomes infinitely close to $a$''.
\end{definition}

What does it mean for $x$ to become ``infinitely close to $a$''? Let's consider for $a=2$. When we say $x\to 2$, this implies some number infinitely close to 2. This can be $1.0000....0001$ or $0.99999....9$; we can clearly see that both of these numbers are ``infinitely'' close to 2 (see Figure \ref{fig:infinitely-close}).  

\begin{figure}[ht]
    \centering
    \incfig{infinitely-close}
    \caption{We keep on dividing the parts between $ \frac{1}{2}$ and 1 by half until infinity. It will never reach 1, but it becomes infinitely close to 1 as we keep on repeating this process.}
    \label{fig:infinitely-close}
\end{figure}

The most important thing to remember for a limit is that it approaches the value, but is never that value. This becomes important when we are determing the derivative of functions using the First Principle.

It should be noted that the value for the limit of a variable is independent to other variables. For example, if we had $\displaystyle \lim_{x\to 0}(x+y)$, the $y$ is independent of the $x$, so we can really write $\displaystyle y+\lim_{x\to 0}x$.\sidenote{The limit rule that we just used is called the sum property for limits, where $y$ is a constant that isn't dependent of $x$.}

\section{Sum Property}\label{def:sum-property}
\begin{definition}
    For functions $f(x)$ and $g(x)$, the limit of their sum is the sum of their limits 
    \[
        \lim_{x\to a} \left[ f(x)+g(x) \right]= \lim_{x\to a}f(x)+ \lim_{x\to a}g(x)
    \]
    And if there is a constant $C$
    \[
        \lim_{x\to a} \left[ f(x)+C \right]= C+\lim_{x\to a}f(x)
    \]
\end{definition}
And we will not dive into the proof for this property. 

\section{Difference Property}
\begin{definition}
    For functions $f(x)$ and $g(x)$, the limit of their difference is the difference of their limits 
    \[
        \lim_{x\to a} \left[ f(x)-g(x) \right]= \lim_{x\to a}f(x)- \lim_{x\to a}g(x)
    \]
    And if there is a constant $C$
    \[
        \lim_{x\to a} \left[ f(x)-C \right]=-C+ \lim_{x\to a}f(x)
    \]
    Or 
    \[
        \lim_{x\to a} \left[ C-f(x) \right]=C- \lim_{x\to a}f(x)
    \]
\end{definition}
\begin{myproof}
    We will prove this using Definition \ref{def:sum-property}
    \begin{align*}
        \lim_{x\to a} \left[ f(x)-g(x) \right]&= \lim_{x\to a}f(x)+ (\lim_{x\to a} -\left[ g(x) \right])\\
                                              &= \lim_{x\to a}f(x)- \lim_{x\to a}g(x)
    \end{align*}
    Where it because the negative is independent of the limit (it is a constant multiplier), we can drag it outside of the limit.\sidenote{These proofs aren't ``mathematical'', but these properties are pretty intuitive so this will suffice. You can search up the actual proofs online. They are pretty weird though.}
\end{myproof}

\newpage
\section{Product Property}\label{def:product}
\begin{definition}
    For functions $f(x)$ and $g(x)$, the limit of their product is the product of their limits 
    \[
        \lim_{x\to a}f(x)g(x)= \lim_{x\to a}f(x)\cdot \lim_{x\to a}g(x)
    \]
\end{definition}
The proof for this is too obscure, so we won't include that here. Just remember this property.

\section{Quotient Property}
\begin{definition}
    For functions $f(x)$ and $g(x)$, the limit of their quotient is the quotient of their limits 
    \[
        \lim_{x\to a} \frac{f(x)}{g(x)}= \frac{ \lim_{x\to a}f(x)}{ \lim_{x\to a}g(x)}
    \]
\end{definition}
\begin{myproof}
    Using Definition \ref{def:product}
    \begin{align*}
        \lim_{x\to a} \frac{f(x)}{g(x)}&= \lim_{x\to a}f(x)[g(x)]^{-1}\\
                                       &= \lim_{x\to a}f(x) \lim_{x\to a}[g(x)]^{-1}\\ 
                                       &= \frac{ \lim_{x\to a}f(x)}{ \lim_{x\to a}g(x)}
    \end{align*}
\end{myproof}

\section{Using Limit Properties}
Now that we have established the properties of limits, we can use them in some examples. 

\begin{fullwidth}
\begin{eg}
    Evaluate $\displaystyle \lim_{h\to 0}(x+h)$. In this case, we see that the limit is evaluating $h\to0$. Therefore, $x$ is completely independent in this limit. Hence, we write
    \begin{align*}
        \lim_{h\to 0}(x+h)&=x+ \lim_{h\to 0}h\\ 
                          &=x
    \end{align*}
    Since as $h\to0$, we approximate it to be 0, leaving us with just $x$.
\end{eg}
\begin{eg}
    Evaluate $\displaystyle \lim_{x\to 2} \frac{x^2}{x+1}$. We can actually just plug in $x=2$ into the numerator and denominator 
    \begin{align*}
        \lim_{x\to 2} \frac{x^2}{x+1}&= \frac{(2)^2}{(2)+1}\\ 
                                     &= \frac{4}{3}
    \end{align*}
\end{eg}
\end{fullwidth}
\newpage 
\begin{marginfigure}
    \centering
    \incfig{exponential-versus-linear}
    \caption{$f(x)=e^x$ clearly outgrows $g(x)=x$ for larger values of $x$.}
    \label{fig:exponential-versus-linear}
\end{marginfigure}
\begin{eg}
    Evaluate $\displaystyle \lim_{x\to \infty } \frac{e^x}{x}$. Now to evaluate this, we need to consider which one rises faster, or the same: $e^x$ or $x$? The reason for this is because if the denominator becomes really big, then this just shrinks to 0, whereas if the numerator is really big, then this grows to infinity. We see that for large values of $x$, $e^x$ clearly outgrows $x$, since it is an exponential function, whereas $x$ is a linear function. So for example, if $x$ was say 99, we will get something like 
    \begin{align*}
        \lim_{x\to 99} \frac{e^x}{x}&=\frac{e^{99}}{99}\\
        &= \text{very large number}
    \end{align*}
    And the result would just evaluate larger and larger for larger values of $x$.
    We can also see in Figure \ref{fig:exponential-versus-linear} that $e^x$ undoubtedly clearly outgrows $x$, so the numertor outgrows the denominator, hence the value shrinks to 0 for $x\to \infty $.
\end{eg}
\begin{fullwidth}
\begin{eg}
    Evaluate $\displaystyle \lim_{x\to 1} \frac{x}{x-1}$. We can actually just see that letting $x\to1$ makes the denominator approach $0$, and so therefore the fraction approaches $ \infty $
    \begin{align*}
        \lim_{x\to 1} \frac{x}{x-1}&\to \frac{1}{1-1}\\ 
                                   &\to \frac{1}{0}\\ 
                                   &\to \infty 
    \end{align*}
    WHere we write it approaches instead of equals, because we technically cannot have 0 in the denominator. For a more general proof, we will first add $1-1$ so that it is 0 in the numerator 
    \[
        \lim_{x\to 1} \frac{x}{x-1}= \lim_{x\to 1} \frac{x-1+1}{x-1}\\
    \]
    Then using the limit of sums property 
    \begin{align*}
        \lim_{x\to 1} \frac{x-1+1}{x-1}&= \lim_{x\to 1} \frac{x-1}{x-1}+ \lim_{x\to 1} \frac{1}{x-1}\\ 
                                       &= \lim_{x\to 1}(1)+ \lim_{x\to 1} \frac{1}{x-1}\\ 
                                       &=1+ \infty \\ 
                                       &= \infty 
    \end{align*}
\end{eg}
\end{fullwidth}
\begin{eg}
    Evaluate $\displaystyle \lim_{x\to -3} \frac{x^3+4x^2+2x-3}{(x-2)(x+3)}$. The trick for this problem (and similar problems) is that we can immediately tell that if this limit evaluates to something, then $(x+3)$ must be factor for the numerator as well.\sidenote{The reason this must be true is because we see that $x\to-3$ makes the factor $(x+3)=0$, and so therefore if we want this limit to not be some value divided by 0, then the numerator must have another $(x+3)$ to cancel out with the $(x+3)$ in the denominator.} And we can immediately see that's the case, since $-3$ is a solution to the numerator, therefore $(x+3)$ must be a factor. 
\end{eg}
\newpage
\begin{fullwidth}
    So if we use synthetic division for the numerator 
    \begin{align*}
        \lim_{x\to -3} \frac{x^3+4x^2+2x-3}{(x-2)(x+3)}&= \lim_{x\to -3} \frac{(x+3)(x^2+x-1)}{(x-2)(x+3)}\\
                                                       &= \lim_{x\to -3}\frac{x^2+x-1}{x-2}\\ 
                                                       &= \frac{(-3)^2+(-3)-1}{(-3)-2}\\ 
                                                       &= \frac{9-4}{-5}\\ 
                                                       &=-1
    \end{align*}
\begin{eg}
    Evaluate $\displaystyle \lim_{x\to 1} \left[ \frac{1}{1-x}- \frac{3}{1-x^3} \right]$. To evaluate this, we will first combine fractions. How do we combine fractions? We will first make use of the difference of cubes identity:
    \begin{align*}
        a^3-b^3&= (a-b)(a^2+b^2+ab)\\
        1-x^3&=(1-x)(x^2+x+1)
    \end{align*}
    Substituting
    \begin{align*}
        \lim_{x\to 1} \left[ \frac{1}{1-x}- \frac{3}{1-x^3} \right]&= \lim_{x\to 1} \left[ \frac{1}{1-x}- \frac{3}{(1-x)(x^2+x+1)} \right]\\
                                                                   &= \lim_{x\to 1} \frac{x^2+x-2}{(1-x)(x^2+x+1)}\\ 
    \end{align*}
    Now we still see that we get an undefined value, namely $ \frac{0}{0}$. What we can do is we can try and factor the numerator so that the $(1-x)$ term cancels out with the denominator. Luckily, $(x-1)$, which is the negative of the $(1-x)$ is a factor of the numerator
    \begin{align*}
        \lim_{x\to 1} \frac{x^2+x-2}{(1-x)(x^2+x+1)}&= \lim_{x\to 1} \frac{(x-1)(x+2)}{(1-x)(x^2+x+1)}\\
                                                    &= \lim_{x\to 1} \frac{-\cancel{(1-x)}(x+2)}{\cancel{(1-x)}(x^2+x+1)}\\ 
                                                    &=- \lim_{x\to 1} \frac{x+2}{x^2+x+1}\\ 
    \end{align*}
    And now we can substitute in $x=1$ because it won't give us an undefined answer 
    \begin{align*}
        - \lim_{x\to 1} \frac{x+2}{x^2+x+1}&= - \frac{1+2}{1^2+1+1}\\
                                           &=-1
    \end{align*}
\end{eg}
\end{fullwidth}
\newpage 
\begin{fullwidth}
\begin{eg}
    Evaluate $\displaystyle \lim_{x\to 4} \frac{ \sqrt{x+5}-3}{x-4}$. We see that if we plug in $x=4$ we get $ \frac{0}{0}$. This is an issue, because this is undefined. However, we have a trick: if we multiply the fraction by $\displaystyle \frac{ \sqrt{x+5}+3}{ \sqrt{x+5}+3}$, the radical in the numerator dissapears 
    \begin{align*}
        \lim_{x\to 4} \frac{ \sqrt{x+5}-3}{x-4}&= \lim_{x\to 4} \frac{ \sqrt{x+5}-3}{x-4} \frac{ \sqrt{x+5}+3}{ \sqrt{x+5}+3}\\ 
                                               &= \lim_{x\to 4} \frac{x-4}{(x-4)( \sqrt{x+5}+3)}\\ 
                                               &= \lim_{x\to 4} \frac{1}{ \sqrt{x+5}+3}\\ 
                                               &= \frac{1}{6}
    \end{align*}
\end{eg}
\begin{eg}
    Evaluate $\displaystyle \lim_{x\to 2} \frac{x^2-2x}{ \sqrt{x+2}-2}$. We will first factor out an $x$ from the numerator
    \[
        \lim_{x\to 2} \frac{x^2-2x}{ \sqrt{x+2}-2}= \lim_{x\to 2} \frac{x(x-2)}{ \sqrt{x+2}-2}
    \]
Then will multiply the fraction by $\displaystyle \frac{ \sqrt{x+2}+2}{ \sqrt{x+2}+2}$ 
    \begin{align*}
        \lim_{x\to 2} \frac{x(x-2)}{ \sqrt{x+2}-2}&= \lim_{x\to 2} \frac{x(x-2)}{ \sqrt{x+2}-2} \frac{ \sqrt{x+2}+2}{ \sqrt{x+2}+2}\\ 
                                                  &= \lim_{x\to 2} \frac{x(x-2)( \sqrt{x+2}+2)}{x+2-4}\\ 
                                                  &= \lim_{x\to 2} \frac{x\cancel{(x-2)}( \sqrt{x+2}+2)}{\cancel{x-2}}\\
                                                  &= 2( \sqrt{2+2}+2)\\ 
                                                  &= 8
    \end{align*}
\end{eg}
\end{fullwidth}
\newpage
\begin{problems}
    \item{Evaluate $\displaystyle \lim_{x\to 1} \left( \frac{1}{x^2-1}- \frac{2}{x^4-1} \right)$.}
    \item{Evaluate $\displaystyle \lim_{x\to 2} \frac{x^2-2x}{ \sqrt{2x}-2}$.}
\end{problems}

\newpage
\begin{solutions}
    \item{We know that $x^4-1=(x^2-1)(x^2+1)$ and then we will combine fractions
            \begin{align*}
                \lim_{x\to 1} \left( \frac{1}{x^2-1}- \frac{2}{x^4-1} \right)&= \lim_{x\to 1} \left( \frac{1}{x^2-1}- \frac{2}{(x^2-1)(x^2+1)} \right)\\ 
                                                                             &= \lim_{x\to 1} \left( \frac{x^2-1}{(x^2-1)(x^2+1)} \right)\\ 
                                                                             &= \lim_{x\to 1} \frac{1}{x^2+1}\\ 
                                                                             &= \frac{1}{2}
            \end{align*}
        }
    \item{We multiply the fraction by $\displaystyle \frac{ \sqrt{2x}+2}{ \sqrt{2x}+2}$ so that the radicals dissapear
            \begin{align*}
                \lim_{x\to 2} \frac{x^2-2x}{ \sqrt{2x-2}}&= \lim_{x\to 2} \frac{x^2-2x}{ \sqrt{2x-2}} \frac{ \sqrt{2x}+2}{ \sqrt{2x}+2}\\ 
                                                         &= \lim_{x\to 2} \frac{x^2 \sqrt{2x}+2x^2-2x \sqrt{2x}-4x}{2x-4}\\ 
                                                         &= \frac{1}{2} \lim_{x\to 2} \frac{2x(x-2)+x^2 \sqrt{2x}-2x \sqrt{2x}}{x-2}\\ 
                                                         &= \frac{1}{2} \lim_{x\to 2} \frac{2x(x-2)+x \sqrt{2x}(x-2)}{x-2}\\ 
                                                         &= \frac{1}{2} \lim_{x\to 2}(2x+x \sqrt{2x})\\ 
                                                         &= \frac{1}{2}(4+4)\\ 
                                                         &= 4
            \end{align*}
        }
\end{solutions}

\end{document}
