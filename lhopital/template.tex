\documentclass[working]{tuftebook}

\usepackage{standalone}
\input{../preamble/preamble.tex}
\input{../preamble/laterPreamble.tex}

\begin{document}
\renewcommand{\thepage}{\arabic{page}}
\pagestyle{plain}
\pagestyle{normal}
\chapter{L'Hopital's Rule}
\vspace{-2em}
Sometimes, there are limits that we cannot evaluate by simply manipulation. For example, how are we suppose to evaluate 
\[
    \lim_{x\to 0} \frac{3^x-1}{x}
\]
The only way we can do this, is by using something called \emph{L'Hopital's Rule}.

\begin{definition}[L'Hopital's Rule]
    Suppose we have two arbitrary functions $f(x)$ and $g(x)$. Then, as $x\to0$ or $x\to\infty$, if both $f(x)=0$ and $g(x)=0$ or $f(x)=\infty$ and $g(x)=\infty$, then 
    \[
        \lim_{x\to 0} \frac{f(x)}{g(x)}= \lim_{x\to 0} \frac{f'(x)}{g'(x)}
    \]
    Or 
    \[
        \lim_{x\to \infty } \frac{f(x)}{g(x)}= \lim_{x\to 0} \frac{f'(x)}{g'(x)}
    \]
    In other words, you differentiate the numerator and denominator.
\end{definition}

\begin{eg}
    Evaluate $\displaystyle \lim_{x\to 0} \frac{3^x-1}{x}$. Differentiating the numerator and denominator 
    \begin{align*}
        \lim_{x\to 0} \frac{3^x-1}{x} &=\lim_{x\to 0} \frac{3^x \ln 3}{1}\\
                                      &= \ln 3 \frac{1}{1}\\ 
                                      &= \ln 3
    \end{align*}
\end{eg}

\newpage 
\begin{challenge}
    \item{Prove that $\displaystyle \lim_{x\to \infty } \left( 1+ \frac{a}{x} \right)^bx=e^{ab}$.}
    \item{Evaluate $\displaystyle \lim_{x\to \infty } \left(  \frac{2x-1}{2x+1} \right)^x$.}
\end{challenge}
\begin{solutions}
    \item{}
    \item{We plug in $x= \infty $ we get $( \frac{ \infty }{ \infty })^ \infty $, which doesn't work. Instead, we will first separate the fraction
            \begin{align*}
                \lim_{x\to \infty } \left( \frac{2x-1}{2x+1} \right)^x&= \lim_{x\to \infty } \left( \frac{2x+1-2}{2x+1} \right)^x\\ 
                                                                      &= \lim_{x\to \infty } \left( 1- \frac{2}{2x+1} \right)^x\\ 
            \end{align*}
        Then we make the substitution $u=2x+1$, $x= \frac{u-1}{2}$, and limits change from $\displaystyle \lim_{x\to \infty }\to \lim_{u\to \infty }$
        \begin{align*}
            \lim_{x\to \infty } \left( 1- \frac{2}{2x+1} \right)^x&= \lim_{u\to \infty } \left( 1- \frac{2}{u} \right)^ \frac{u-1}{2}\\ 
                                                                  &= \lim_{u\to \infty } \left[ \frac{ \left( 1- \frac{2}{u} \right)^u}{ \left( 1- \frac{2}{u} \right)} \right]^ \frac{1}{2}
        \end{align*}
        Then, the denominator evaluates to $\displaystyle \lim_{u\to \infty } \left( 1- \frac{2}{u} \right)=0$
        \[
            \lim_{u\to \infty } \left[ \frac{ \left( 1- \frac{2}{u} \right)^u}{ \left( 1- \frac{2}{u} \right)} \right]^ \frac{1}{2}= \lim_{u\to \infty } \left[ \left( 1- \frac{2}{u} \right)^u \right]^ \frac{1}{2}\\
        \]
        And from the result obtained in the first challenge problem, we know that $\displaystyle \lim_{u\to \infty } \left( 1- \frac{2}{u} \right)^u= \lim_{u\to \infty } \left( 1+ \frac{-2}{u} \right)^u=e^{-2}$ 
        \begin{align*}
            \lim_{u\to \infty } \left[ \left( 1- \frac{2}{u} \right) \right]^ \frac{1}{2}&= \left[ e^{-2} \right]^ \frac{1}{2}\\
                                                                                         &= \frac{1}{e} 
        \end{align*}
        }
\end{solutions}

\end{document}
